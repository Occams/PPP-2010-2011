\documentclass{article}
\usepackage[T1]{fontenc}
\usepackage[ngerman,english]{babel}
\usepackage[utf8]{inputenc}
\usepackage[vlined,ruled]{algorithm2e}
\usepackage{libertine,calc,microtype,parskip,lipsum,booktabs,textcomp,csquotes,
	enumerate,amssymb,vmargin,fancyhdr,fixltx2e,makeidx,listings,ellipsis,remreset,xcolor,lastpage,caption,fancybox,verbatim,amsmath}
\usepackage{graphicx}
\usepackage[pdftex]{hyperref}
\usepackage{amsmath}

% Title
\def\thetitle{Praktische Parallelprogrammierung --- Blatt 04}

% ------------------------------------------
% -------- xcolor - (Tango)
% ------------------------------------------

\definecolor{LightButter}{rgb}{0.98,0.91,0.31}
\definecolor{LightOrange}{rgb}{0.98,0.68,0.24}
\definecolor{LightChocolate}{rgb}{0.91,0.72,0.43}
\definecolor{LightChameleon}{rgb}{0.54,0.88,0.20}
\definecolor{LightSkyBlue}{rgb}{0.45,0.62,0.81}
\definecolor{LightPlum}{rgb}{0.68,0.50,0.66}
\definecolor{LightScarletRed}{rgb}{0.93,0.16,0.16}
\definecolor{Butter}{rgb}{0.93,0.86,0.25}
\definecolor{Orange}{rgb}{0.96,0.47,0.00}
\definecolor{Chocolate}{rgb}{0.75,0.49,0.07}
\definecolor{Chameleon}{rgb}{0.45,0.82,0.09}
\definecolor{SkyBlue}{rgb}{0.20,0.39,0.64}
\definecolor{Plum}{rgb}{0.46,0.31,0.48}
\definecolor{ScarletRed}{rgb}{0.80,0.00,0.00}
\definecolor{DarkButter}{rgb}{0.77,0.62,0.00}
\definecolor{DarkOrange}{rgb}{0.80,0.36,0.00}
\definecolor{DarkChocolate}{rgb}{0.56,0.35,0.01}
\definecolor{DarkChameleon}{rgb}{0.30,0.60,0.02}
\definecolor{DarkSkyBlue}{rgb}{0.12,0.29,0.53}
\definecolor{DarkPlum}{rgb}{0.36,0.21,0.40}
\definecolor{DarkScarletRed}{rgb}{0.64,0.00,0.00}
\definecolor{Aluminium1}{rgb}{0.93,0.93,0.92}
\definecolor{Aluminium2}{rgb}{0.82,0.84,0.81}
\definecolor{Aluminium3}{rgb}{0.73,0.74,0.71}
\definecolor{Aluminium4}{rgb}{0.53,0.54,0.52}
\definecolor{Aluminium5}{rgb}{0.33,0.34,0.32}
\definecolor{Aluminium6}{rgb}{0.18,0.20,0.21}
\definecolor{Brown}{cmyk}{0,0.81,1,0.60}
\definecolor{OliveGreen}{cmyk}{0.64,0,0.95,0.40}
\definecolor{CadetBlue}{cmyk}{0.62,0.57,0.23,0}

% ------------------------------------------
% -------- vmargin
% ------------------------------------------

%\setmarginsrb{hleftmargini}{htopmargini}{hrightmargini}{hbottommargini}%{hheadheighti}{hheadsepi}{hfootheighti}{hfootskipi}
\setpapersize{A4}
\setmarginsrb{3cm}{1cm}{3cm}{1cm}{6mm}{7mm}{5mm}{15mm}

% ------------------------------------------
% -------- fancyhdr
% ------------------------------------------
%\fancyheadoffset[L]{\marginparsep+\marginparwidth}
\fancyhf{}
\fancyhead[L]{\bfseries{\nouppercase{\thetitle}}}
\fancyhead[R]{\bfseries{Seite \thepage\ von \pageref{LastPage}}}
\renewcommand{\headrulewidth}{0.5pt}
\renewcommand{\footrulewidth}{0pt}
\fancypagestyle{plain}{
\fancyhf{}
\fancyfoot[R]{\bfseries{Seite \thepage\ von \pageref{LastPage}}}
\renewcommand{\headrulewidth}{0pt}
\renewcommand{\footrulewidth}{0pt}
}

% ------------------------------------------
% -------- hyperref
% ------------------------------------------

\hypersetup{
	%breaklinks=true,
	pdfborder={0 0 0},
	bookmarks=true,         % show bookmarks bar?
	unicode=false,          % non-Latin characters in Acrobat’s bookmarks
	pdftoolbar=true,        % show Acrobat’s toolbar?
	pdfmenubar=true,        % show Acrobat’s menu?
	pdffitwindow=true,     % window fit to page when opened
	pdfstartview={FitH},    % fits the width of the page to the window
	pdftitle={Komplexitätstheorie Übung},    % title
	pdfauthor={Huber Bastian},     % author
    pdfsubject={Übungsblatt},   % subject of the document
    pdfcreator={Huber Bastian},   % creator of the document
    pdfproducer={Huber Bastian}, % producer of the document
    pdfkeywords={Komplexitätstheorie, Passau}, % list of keywords
    pdfnewwindow=true,      % links in new window
    colorlinks=true,       % false: boxed links; true: colored links
    linkcolor=black,          % color of internal links
    citecolor=black,        % color of links to bibliography
    filecolor=magenta,      % color of file links
    urlcolor=DarkSkyBlue           % color of external links
}

% ------------------------------------------
% -------- listings
% ------------------------------------------
 
\lstset{
		breakautoindent=true,
		breakindent=2em,
		breaklines=true,
		tabsize=4,
		frame=blrt,
		frameround=tttt,
		captionpos=b,
		basicstyle=\scriptsize\ttfamily,
		keywordstyle={\color{SkyBlue}},
		%commentstyle={\color{OliveGreen}},
		stringstyle={\color{OliveGreen}},
		showspaces=false,
		%numbers=right,
		%numberstyle=\scriptsize,
		%stepnumber=1, 
		%numbersep=5pt,
		%showtabs=false
		prebreak = \raisebox{0ex}[0ex][0ex]{\ensuremath{\hookleftarrow}},
		aboveskip={1.5\baselineskip},
		columns=fixed,
		upquote=true,
		extendedchars=true
}
\fontsize{3mm}{4mm}\selectfont

% ------------------------------------------
% -------- misc
% ------------------------------------------
\newcommand{\bibliographyname}{Bibliography}
\setcounter{secnumdepth}{3}
\setcounter{tocdepth}{3}
\clubpenalty = 10000
\widowpenalty = 10000
\displaywidowpenalty = 10000
\setlength\fboxsep{6pt}
\setlength\fboxrule{1pt}
\renewcommand*\oldstylenums[1]{{\fontfamily{fxlj}\selectfont #1}}
%% Set table margins.
{\renewcommand{\arraystretch}{2}
\renewcommand{\tabcolsep}{0.4cm}}

% ------------------------------------------
% -------- hyphenation rules
% ------------------------------------------
\hyphenation{}

\author{Bastian Huber\\(51432) \and Daniel Watzinger\\(51746)}
\title{\textbf{\huge{\thetitle}}\\\Large\textsc{}\\\large\textsc{}}
\date{\today}

\begin{document}

% Specify hyphenation rules.
\hyphenation{}

\maketitle

\pagestyle{fancy}

\section*{Aufgabe 4}
\begin{enumerate}[a)]
	\item
	Die n-Körper Simulation wurde sequentiell und parallel in drei Versionen implementiert. Zum einen existiert eine reine OpenMP Lösung zum anderen zwei verschiedene OpenMP+MPI Lösugen. Eine der beiden OpenMP+MPI Lösungen nutzt Newtons drittes Gesetz global, d.h. macht möglichst wenig Berechnungen, kommuniziert dafür mehr. Die andere macht dafür redundante Berechnungen, kommuniziert dafür weniger.
	
	\textit{Messen Sie Berechnungszeit, Speedup und Interaktionsrate mit dem Beispiel spiralgalaxie.dat.}
	
	\begin{center}
		\begin{tabular}{llllllllll}
			 	        	& Berechnungszeit & Speedup (Vgl. zu squentiell) & Interaktionsrate \\
			 SEQUENTIAL 	& 68.188415 s		  & 1		& 23640672.686121  \\
			 8 OMP only 		& 15.853464 s		  & 4.30		& 101682509.261535 \\
			 4 MPI * 2 OMP	& 30.720369 s		  &	2.22		& 52473979.072342	\\
			 4 MPI * 2 OMP (global Newton)	& 15.066842		  &	4.53		& 106991232.801139	\\
			 6 MPI * 2 OMP	& 19.870298	s	  &	3.43		& 81127117.469287	\\
			 6 MPI * 2 OMP (global Newton)	& 9.573131 s		  &	7.12		& 168390049.190081	\\		 
		\end{tabular}
		\captionof{table}{Berechnungszeit, Speedup und Interaktionsrate (Steps 10000, $\Delta_t$ 31600000000) }
		\label{tab:}
	\end{center}
	
	\item[e)]
	Indem ein Verlust an Genauigkeit in Kauf genommen wird, kann man die Berechnungsschritte
	und somit die benötigte Rechendauer reduzieren.
	\begin{itemize}
		\item Einführung eines Thresholds: Für eine neue Schwellwertfunktion muss folgendes
	gelten Threshold(i, j) =  Threshold(j, i). In die Funktion sollten die Massen der beiden
	Körper i und j sowie deren Distanz relativ zueinander eingehen.
	Falls Threshold(i,j) < Schwellwert dann wird die Berechnung der Kraft die der Körper
	i auf j ausübt übersprungen. Diese Kraft
	wird also als vernachlässigbar eingestuft.
	
		\item Betrachte einen Körper i: Teile den Raum in ein Gitternetz auf. In einer Zelle des Gitters
	können mehrere Körper liegen. Körper die sehr weit entfernt sind von i und in einer Zelle liegen
	werden nun zu einem massiven Körper zusammengefasst. Für diesen fiktiven Körper gilt:
		\begin{itemize}
			\item Position = Schwerpunkt der Körper in der Zelle
			\item Masse = Summe der Massen der Körper in der Zelle
		\end{itemize}
	Nun wird nur die Kraft die der fiktive Körper auf i ausübt berechnet (und umgekehrt).
	\end{itemize}
\end{enumerate}

\end{document}
